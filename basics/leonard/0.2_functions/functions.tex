\documentclass{article}
\usepackage{graphicx}
\usepackage{amsmath}
\usepackage{enumerate}
\usepackage{amsfonts}

\begin{document}

\title{Lecture 0.2: Functions}
\author{Professor Leonard}
\maketitle

This lecture is all about a review of functions.


\section{Functions}
The first big point he's making is that in order for something to be a function, each
input $X$ needs to be associated with only one output $f(X)$.

\subsection{Special Functions}

Need to be careful with something like the following:

\begin{align}
    x^2 + y^2 = 25\\
    y^2 = 25 - x^2\\
    y = \pm \sqrt{25 - x^2}
\end{align}

This isn't a function because there can always be two values (plus or minus) that satisfy
$y^2$.\\

\subsubsection{Piecewise functions}

One of the most common functions is the absolute value function, which is an example of a
\emph{piecewise function}:

\begin{equation}
    \lvert x \rvert = 
    \begin{cases}
        x & \text{if } x \geq 0\\
        -x & \text{if } x < 0
    \end{cases}
\end{equation}

\subsubsection{Domain and range}

\textbf{Domain:} All input values for a function.\\ 
\textbf{Range:} All output values for a function.\\

There can be physical restraints (eg, no negative distance), or formulaic restraints (eg,
$1/x~x \neq 0$.

\textbf{Natural Domain:} All values that 'work' in the function.\\

Example:


\begin{align*}
f(x) = x^3\\
x \in \mathbb{R}\\
\\
g(x) = \frac{1}{(x-1)(x-3)}\\
x \neq 1,~x \neq 3
\end{align*}

Need to watch out for denominators and roots.\\

Left off around 50 min.

\end{document}

